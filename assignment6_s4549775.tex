\documentclass[12pt, a4paper]{article}

\setlength\parskip{1em}
\setlength\parindent{0em}

\title{Assignment 6}

\author{Hendrik Werner s4549775}

\begin{document}
\maketitle

\begin{itemize}
	\item I want to publish my article in the "Journal of Computer Science" (http://thescipub.com/journals/jcs).

	In this journal a classical academic layout is customary. This includes an abstract, an introduction, a body, a conclusion, and then acknowledgements, references, etc.
	\item The goal I want to achieve is getting people interested in computer vision and robotics; and to provide a knowledge basis on which they can further their research into the subject should they be interested.
	\item The target audience are computer scientists in general. I can assume that they are familiar with basic computer science concepts and terminology. My article is supposed to introduce people into computer vision and its applications ins robotics and should therefore not assume a deep familiarity with any concepts and terms specific to this area.
	\item Computer vision and robotics are important subjects of ongoing research in the current computer science landscape. Many computer scientists want to stay up to date, or simply learn it for hobby projects, or for the fun of it. The barrier of entry in this area can be very high however. People will want to read my article, because it serves as an easy to understand introduction into the subject that will provide a basis for further development.
\end{itemize}

\section*{Introduction}
For a university project I wanted to detect the position of a robot and which way it was facing, as well as the position of a ball, using visual input. I looked into computer vision frameworks and eventually chose OpenCV. I expected there to be some prebuilt functionality for something as mundane as detecting an arrow but was disappointed.

In this article I want to present the approach I took to detect the features I needed; and how this was applied in a robotics project. This article is positioned as an introduction into image detection and does not assume familiarity with the subject. Concepts and terminology are introduced as needed;
it serves as a quick overview over image detection and OpenCV in particular.

\section*{Structure}
\begin{enumerate}
	\item Abstract
	\item Introduction
	\item Problem
	\item Further Reading
	\item Conclusion
	\item References
\end{enumerate}

\subsection*{Reasoning}
Because I want to publish in the Journal of Computer Science, I already have this predefined structure:

\begin{enumerate}
	\item Abstract
	\item Introduction
	\item \dots
	\item Conclusion
	\item References
\end{enumerate}

Between the introduction and conclusion I am free to choose my own structure.
\end{document}
